% Intègre le code d'un fichier d'exercice. Il faut donner le répertoire dans lequel se trouve le code puis le nom du fichier.
\newcommand{\inputexcode}[2]{
\subsubsection{#2}
\lstinputlisting{../Exercices/#1/#2}
}

\lstset{
  literate=
  {á}{{\'a}}1 {é}{{\'e}}1 {í}{{\'i}}1 {ó}{{\'o}}1 {ú}{{\'u}}1
  {Á}{{\'A}}1 {É}{{\'E}}1 {Í}{{\'I}}1 {Ó}{{\'O}}1 {Ú}{{\'U}}1
  {à}{{\`a}}1 {è}{{\'e}}1 {ì}{{\`i}}1 {ò}{{\`o}}1 {ù}{{\`u}}1
  {À}{{\`A}}1 {È}{{\'E}}1 {Ì}{{\`I}}1 {Ò}{{\`O}}1 {Ù}{{\`U}}1
  {ä}{{\"a}}1 {ë}{{\"e}}1 {ï}{{\"i}}1 {ö}{{\"o}}1 {ü}{{\"u}}1
  {Ä}{{\"A}}1 {Ë}{{\"E}}1 {Ï}{{\"I}}1 {Ö}{{\"O}}1 {Ü}{{\"U}}1
  {â}{{\^a}}1 {ê}{{\^e}}1 {î}{{\^i}}1 {ô}{{\^o}}1 {û}{{\^u}}1
  {Â}{{\^A}}1 {Ê}{{\^E}}1 {Î}{{\^I}}1 {Ô}{{\^O}}1 {Û}{{\^U}}1
  {œ}{{\oe}}1 {Œ}{{\OE}}1 {æ}{{\ae}}1 {Æ}{{\AE}}1 {ß}{{\ss}}1
  {ç}{{\c c}}1 {Ç}{{\c C}}1 {ø}{{\o}}1 {å}{{\r a}}1 {Å}{{\r A}}1
  {²}{{\up{2}}}1 {€}{{\euro}}1 {£}{{\pounds}}1 {°}{{\up{o}}}1,
  backgroundcolor=\color{beige},
  basicstyle=\ttfamily\footnotesize,
  breaklines=true,
  commentstyle=\color{lightgreen}\itshape,
  extendedchars=true,
  frame=lRtB,
  frameround=ffft,
  keywordstyle=\color{blue}\bfseries,
  numbers=left,
%  stepnumber=3,
  stringstyle=\color{red},
  tabsize=2,
  language=C
}